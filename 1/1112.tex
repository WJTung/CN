\documentclass[12pt, a4paper]{article}
\usepackage{fontspec}
\usepackage{xeCJK}
\usepackage{hyperref}
\setCJKmainfont{微軟正黑體}
\XeTeXlinebreaklocale "zh"
\XeTeXlinebreakskip = 0pt plus 1pt
\usepackage{enumerate}
\usepackage{graphicx}
\date{}
\title{\vspace{-3.0cm} Computer Networks HW1 Report \\ \vspace{0.5cm}}
\author{\normalsize B03902062 資工三 \hspace{0cm} 董文捷}
\begin{document}
\maketitle
In this homework, I choose \texttt{C} as programming language. Although coding in \texttt{C} takes me about 800 lines in total, which seems to be much more time-consuming compared to the convenient \textit{string} functions and \textit{try and except} of \texttt{Python}, \texttt{C} is generally 10 times faster during run time. To make code easier to manage, I seperate different function into four \texttt{.c} files, \texttt{irc.c}, \texttt{cal.c}, \texttt{weather.c}, and \texttt{main.c}. Those \texttt{.c} files can be compiled together to get \texttt{irc\_robot} with \texttt{make}. For better modifiability, different parameters are defined in corresponding \texttt{.h} file, which can be easily adjusted according to various situations. The following are some noteworthy parts of my code.

\begin{enumerate}[1.]

\item
{\bf @cal} \\
To implement calculator, I modify my code written for \textit{Data Structures and Algorithms, Spring 2015}. Nevertheless, the original code is implemented in \texttt{C++} with \textit{stack}, which is not available in \texttt{C}, I use \textit{array} to simulate \textit{stack} instead. The most tricky part is error handling for calculator, calculation of wrong format of expression may cause core dumped and stop the program. To avoid this situation, 
four kind of errors are taken into consideration, \texttt{UNMATCHED\_PARENTHESES}, \texttt{UNEXPECTED\_CHARACTER}, \\ \texttt{DIVIDE\_ZERO}, and \texttt{ILLEGAL\_EXPRESSION}. The program will check possible errors before actually doing the calculation, once an error is found, the calculation will stop, and error type is set properly.
 
\item
{\bf Bonus} \\
For the bonus \textit{HTML Parser} part, {\bf @weather} is implemented by creating a socket for the TCP protocol, then get request to Central Weather Bureau, finally find out weather information in the HTML content of the reply. 

\item
{\bf References}
\begin{itemize}
\item \href{http://www.the-tech-tutorial.com/simple-c-irc-bot-template/}{Simple C++ IRC Bot Template}
\item \href{https://github.com/Themaister/simple-irc-bot}{Themaister/simple-irc-bot}
\item \href{http://coding.debuntu.org/c-linux-socket-programming-tcp-simple-http-client}{C: Linux Socket Programming, TCP, a simple HTTP client}
\end{itemize}

\end{enumerate}
\end{document}

 